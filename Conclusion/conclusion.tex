\section{Conclusion}

To conclude, the room-temperature single-photon emission from defects in hBN under non-resonant excitation was investigated to assess the material's suitability for quantum photonic technology applications. Spectra from various emitters were presented, each exhibiting a sharp ZPL and a weaker PSB, with an average ZPL linewidth of 1.9~THz. Measured decay lifetimes were all on the same order of magnitude, with an average spontaneous decay rate of 0.350~GHz. A specific emitter was fully characterised in terms of its brightness, single-photon purity, and indistinguishability, which were found to be 0.005\%, 0.89, and 0.0075\%, respectively. These results indicate that room-temperature hBN defect emitters are promising for photonic applications that do not rely on photon self interference, such as the BB84 QKD protocol or quantum random number generation.

Future experiments should focus on coupling these single-photon emitters into optical cavities, such as open Fabry–Pérot, to enhance the radiative decay rate via the Purcell effect. This approach would improve both the brightness and indistinguishability of the emission, thereby increasing hBN’s potential for integration into scalable photonic technologies. Furthermore, investigating the temperature dependence of the dephasing rate would provide further insight into the coherence properties of these emitters.
