\section{Experimental Methods}

\begin{itemize}
    \item Setup: Micro-photoluminescence in confocal configuration 
    \begin{itemize}
        \item Lasers (CW or pulsed), and their characterisation. Discuss elliptical Gaussian beam and optimisation program
        \item Detectors (spectrometer + CCD or APD's)
        \item Non-resonant excitation (Carlos' review)
    \end{itemize}
    \item Imaging techniques
    \begin{itemize}
        \item Wide non-resonant excitation of sample + CCD
        \item APD scans with fiber
        \item Coordinate system to locate defects
    \end{itemize}
    \item Cavity coupling 
    \begin{itemize}
        \item Cavity system + techniques 
        \item Q-factor, mesas, micro-mirrors \& lenses
        \item k-space imaging 
        \item White light reflectivity
    \end{itemize}
    \item Lifetime measuremeants and how they are done
    \item $g^{(2)}(0)$ and HBT measurements
\end{itemize}

The main laser used for the excitation of the hBN defects was the LDH-I-B-450-M 459nm laser from PicoQuant. It has both pulsed and continuous mode settings and the laser power can be adjusted via a provided UI or python.