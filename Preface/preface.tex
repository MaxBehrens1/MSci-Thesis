\begin{center}
    \Large{\textbf{Preface}} \\
\end{center}
~\\[0.5 cm]


The work presented in this thesis was carried out as part of a large international collaboration known as COMPHORT, which launched in July 2024. The project brings together research groups from Spain, Germany, the United Kingdom, and Turkey, with the aim of developing a simple, easy-to-use single-photon source device that operates at room temperature to harness the power of quantum communications. For more information, visit the project website: \href{https://quantera.eu/comphort/}{quantera.eu/comphort/}.

This work was carried out as a collaborative effort alongside Juan V. Vidal Martínez-Pons (PhD student), Adolfo Menéndez Rúa (BSc student), Alejandro Izquierdo-Molina (BSc Student), and Sang Kyu Kim (visiting PhD student from the group of Prof. M\"uller, TUM). I would particularly like to thank Juan for his consistent guidance and for working closely with me throughout the year. I am also grateful to Alejandro and Sang Kyu for setting up the Michelson interferometer, and to Sang Kyu for his support in the characterisation of the APDs. I would further like to credit Alejandro for creating the reconfigurable bandpass filter used in the setup, and for producing the defect location maps. I would like to thank Adolfo for his assistance with the visibility measurement data analysis and his general positive attitude in the lab. I would further like to credit the previous year's bachelor students for creating the APD scanning code discussed in section \ref{APD-scan}. All other aspects of the work not specifically mentioned here were carried out either entirely by myself or jointly with Juan. This includes the automation of all measurements, data analysis, and the experimental setup. I would further like to express my gratitude to my supervisor, Dr. Carlos Antón-Solanas, for his exceptional guidance, support, and encouragement throughout the course of this project. Their contributions were invaluable to the success of this project.

The results presented in this thesis have been included in a manuscript under review and are available as a preprint on Arxiv \cite{Martinez-Pons2025}.

All the code created during this project can be found at:

\href{https://github.com/JuanVidalMPons/QOSS}{github.com/JuanVidalMPons/QOSS}. 


\newpage