\begin{center}
    \Large{\textbf{Abstract}} \\
\end{center}
~\\[0.5 cm]

\textbf{English:}

Hexagonal Boron Nitride (hBN) defects offer a promising platform as single-photon sources at room temperature, ideal for integration into scalable quantum photonic technologies. This work presents a comprehensive characterisation of emission from a colour centre in a hBN nanocrystal, presenting the purity, brightness, and indistinguishability under various excitation and filtering conditions. Confocal microscopy measurements yielded an average spontaneous decay rate of 0.350 GHz, far slower than the average spectrum linewidth of 1.9 THz, highlighting strong phonon dephasing. Hanbury Brown and Twiss interferometry reveals a single-photon purity of 0.89 and characteristic blinking dynamics indicating the presence of a third 'dark-level'. Michelson interferometry measures a visibility decay rate corresponding to an indistinguishability of 0.0075\%, confirming the dominance of phonon interactions at room temperature. These results indicate that hBN-based emitters are well suited to applications that do not rely on photon interference, such as many quantum communication protocols.

~\\[2cm]

\textbf{Spanish:}

Los defectos en nitruro de boro hexagonal (hBN) ofrecen una plataforma prometedora como fuentes de fotones individuales a temperatura ambiente, ideales para su integración en tecnologías fotónicas cuánticas escalables. Este trabajo presenta una caracterización completa de la emisión de un centro de color en un nanocristal de hBN, mostrando la pureza, el brillo y la indistinguibilidad bajo diversas condiciones de excitación y filtracion. Las mediciones mediante microscopía confocal producen una tasa media de decaimiento espontáneo de 0.350 GHz, muy inferior al ancho de banda espectral medio de 1.9 THz, lo que evidencia una fuerte descoherencia por fonones. La interferometría de Hanbury Brown y Twiss revela una pureza de 0.89 y dinámicas características de intermitencia que indican la presencia de un tercer `nivel oscuro'. La interferometría de Michelson mide una tasa de decaimiento de la visibilidad que corresponde a una indistinguibilidad de 0.0075\%, confirmando el dominio de las interacciones con fonones a temperatura ambiente. Estos resultados indican que los emisores basados en hBN son muy adecuados para aplicaciones que no dependen de la interferencia de fotones, como muchos protocolos de comunicación cuántica.

\newpage