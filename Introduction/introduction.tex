\section{Introduction}

\begin{itemize}
    \item Why do we need single photons for quantum technologies 
    \begin{itemize}
        \item Communication (QKD)
        \item Sensing
        \item Metrology
    \end{itemize}
    \item Probabilistic vs. deterministic single photon sources
    \begin{itemize}
        \item $\chi^{(2)}$ in SPDC
        \item FWM
        \item Atoms (artificial \& natural)
    \end{itemize}
    \item Photonic trinity: PIB
    \begin{itemize}
        \item P: $g^{(2)}(t)$ \& HBT
        \item I: $T_2$, photon coalescence, HOM effect
        \item B: Purcell effect, cavity, reduced $T_1$  
    \end{itemize}
    \item hBN Description
    \begin{itemize}
        \item Discuss $T_1$ times seen experimentally
        \item Quantum efficiency paramter $\eta_{QE}=\frac{\gamma_{rad}}{\gamma_{rad} + \gamma_{non-rad}}$
        \item Debye-Weller effect $\eta_{DW}=\frac{I_{ZPL}}{I_{ZPL}+I_{PSB}}$
        \item Spectral purity $\Delta \nu \propto \frac{1}{T_2}=\frac{1}{T_2^*}+\frac{1}{2T_1}$. $T_2$ is the coherence time, $T_2^*$ is the pure dephasing rate and $T_1$ is the transition lifetime.
        \item Discuss $T_2$ measured experimentally i.e. Aymeric Delteil and Clarisse Fournier
        \item How to generate these defects (Annealing or Irradiation)
        \item Categorising defects (internet hBN-info)
    \end{itemize}
\end{itemize}

This work investigates the non-resonant excitation of hexagonal boron nitride (hBN) defects at room temperature to achieve bright and pure single-photon emission. Additionally, Purcell enhancement is pursued by coupling the single-photon modes to an open plano-concave Fabry-Pérot cavity.

Efficient single-photon production is critical for advancing quantum technologies, with applications ranging from enhancing measurement precision beyond classical limits \cite{Nagata2007, Vitelli2010} to enabling reliable transport and processing of analogue information in photonic circuits \cite{Bogaerts2020}. Among these applications, quantum key distribution (QKD) is the most mature, leveraging the quantum properties of single photons to detect eavesdropping attempts and create secure communication channels. 

The first QKD protocol proposed was BB84 \cite{Bennett2014}, in which Alice (the sender) transmits photons polarised in one of two bases: the rectilinear basis ($\ket{H}$, and $\ket{V}$) or the diagonal basis ($\ket{D}=(\ket{H}+\ket{V})/\sqrt{2}$, and $\ket{A}=(\ket{H}-\ket{V})/\sqrt{2}$). These states encode binary bits (0 or 1) in two mutually unbiased bases. For each photon, Bob (the receiver) randomly selects a basis for measurement (either rectilinear or diagonal) and, after detection, publicly announces which basis was used, without revealing the measured bit value. Alice and Bob then retain only the results where they chose the same basis, using them to construct a shared secret key.

If an eavesdropper (Eve) intercepts and measures a photon, she must randomly select a basis, with a $ \frac{1}{2} $ probability of choosing incorrectly. When she measures in the wrong basis, the quantum state collapses into an incorrect polarization. If she resends the photon, it remains in the wrong state with probability $ \frac{1}{2} $, introducing an error in Bob’s measurement. Alice and Bob detect Eve by comparing $ n $ bits of their shared key. The probability of Eve measuring each photon in the correct basis by chance is $\left(\frac{1}{2}\right)^n$, which decreases exponentially with $n$, allowing security to be arbitrarily strengthened.

While BB84 is provably secure against eavesdropping in an idealised quantum setting \cite{Shor2000}, a significant weakness arises when multi-photon pulses are used instead of true single-photon sources. In such cases, Eve can exploit the photon-number splitting (PNS) attack by using a beam splitter to measure one of the transmitted photons while allowing the other to continue to Bob undisturbed. This allows her to extract information about the key without introducing detectable errors, highlighting the necessity of true single-photon sources for ensuring secure quantum communication.

\subsection{Benchmarking Single-Photon Sources}

An ideal single-photon source should emit exactly one photon on demand, with each photon possessing identical characteristics. To assess the quality of a single-photon source, three key parameters are evaluated: Purity ($P$), Indistinguishability ($I$), and Brightness ($B$). 

\subsubsection{Purity}
Purity is a loss-independent measure that quantifies the likelihood that the source emits only one photon at a time. It is defined as $P = 1 - g^{(2)}(0)$, where $g^{(2)}(0)$ is the second-order correlation function at zero time delay. The function $g^{(2)}(\tau)$ describes the statistical correlation between detecting two photons separated by a time delay $\tau$ relative to a completely random (Poissonian) photon distribution:

\begin{equation}
    g^{(2)}(\tau) = \frac{\langle I(t)I(t+\tau)\rangle}{\langle I(t)\rangle^2}
\end{equation}

where $\langle \rangle$ denotes a time-averaged quantity, and $I(t)$ is the measured photon intensity at time $t$. For an ideal single-photon emitter, $g^{(2)}(0) = 0$, indicating perfect antibunching, meaning that photons are emitted strictly one at a time, with no possibility of multi-photon events. In contrast, coherent light (such as from a laser) follows Poissonian statistics and has a $g^{(2)}(0) = 1$, which indicates no correlation between photon detections. A value of $g^{(2)}(0) > 1$ is characteristic of bunched light, such as thermal sources, where photons tend to arrive in clusters due to bosonic bunching effects. In practice, achieving $g^{(2)}(0) < 0.5$ is a strong indicator of a non-classical light source, supporting potential single-photon emission. This measurement is typically performed using a Hanbury Brown and Twiss (HBT) interferometer \cite{Brown1956}, which will be discussed in detail in \textbf{Section X}. 

\subsubsection{Indistinguishability}

Indistinguishability, also a loss-independent measure, quantifies the likelihood that two photons will interfere in a Hong-Ou-Mandel (HOM) setup \cite{Hong1987}. The Hong-Ou-Mandel effect states that when two indistinguishable photons arrive simultaneously at the input ports of a 50:50 beam splitter, they will always exit together through the same output port, rather than taking separate paths \cite{Hong1987}.

For two photons to be indistinguishable, they must occupy the same spatial mode, have identical spectral and temporal profiles, and be in equivalent polarisation and phase states \cite{Senellart2017}. The spatial mode refers to the spatial distribution of the photon wavepacket; typically, single-mode optical fibres are used to ensure that photons are collected into the same well-defined optical mode. Spectral indistinguishability is determined by the line-width of the emitter, while the temporal profile is governed by the emitter’s lifetime, $T_1$. Polarisation indistinguishability requires that the photons have the same polarisation orientation, which can be ensured using polarisers and wave plates. The phase coherence of single photons is influenced by the dephasing rate, given by $1/T_2^*$, which can be measured using a Michelson interferometer \cite{Michelson1887}. A further explanation of the Michelson interferometer and the measurement of $T_2^*$ will be provided in \textbf{Section X}. In general the coherence time, $T_2$, is governed by both the lifetime and dephasing rate, given by:

\begin{equation}
    \frac{1}{T_2}=\frac{1}{T_2^*} + \frac{1}{2T_1}.
\end{equation}

The coherence time $T_2$ determines the timescale over which a quantum emitter produces indistinguishable photons. It is this coherence time that sets the spectral linewidth of the emitter, as governed by the energy-time uncertainty principle: a shorter $T_2$  corresponds to a broader emission linewidth.

The degree of indistinguishability is typically assessed by measuring the $g^{(2)}(0)$, often denoted $g^{(2)}_{\text{HOM}}(0)$, at the output of a path-unbalanced Mach-Zehnder interferometer. One arm of the interferometer includes a delay line matched to the temporal separation between sequential photons. The temporal alignment of the photons can be verified using a pulsed laser, typically with a pulse width on the order of 3 ps, which is much shorter than the temporal envelope of the single-photons (usually several nanoseconds at room temperature). One of the output ports of the second beam splitter is directed to a spectrometer to observe the interference. When the photons are not temporally overlapped, a spectral beating pattern is visible. In contrast, when they are perfectly aligned in time, the interference produces a Gaussian-shaped spectral feature. 

The indistinguishability can be approximately expressed as $ I \approx 1 - 2g^{(2)}_{\text{HOM}}(0) $. This relation is only approximate because it assumes the source emits exactly one photon per pulse. In practice, multi-photon emission and other imperfections can affect the measured indistinguishability. The expression becomes exact in the ideal case of a perfect single-photon source with no multi-photon contributions. The factor of two arises from the fact that, for a pure but fully distinguishable photon source, $ g^{(2)}_{\text{HOM}}(0) = 0.5 $, as the two photons have a 50\% probability of exiting the same output port of the beam splitter due to the absence of quantum interference. 


\subsubsection{Brightness}

The brightness describes the probability that a single photon is emitted per excitation pulse. Unlike $P$ and $I$, brightness is a loss-dependent measure, as it depends on the detection efficiency and optical setup. Therefore, when comparing sources, it is important to distinguish between the brightness at the source ($B_s$), measured before any optical losses, and the detected brightness ($B_d$), which is affected by transmission losses, collection efficiency and detector efficiency. For meaningful comparisons, experimental setups must carefully characterise and account for losses throughout the optical system.

\subsubsection{Relevance to Quantum Technologies}
Purity, indistinguishability, and brightness all take values between 0 and 1, with an ideal single-photon source achieving $P = I = B = 1$. In reality it is challenging to optimise all three values a balance is required depending on the application:

\begin{itemize}
    \item Quantum cryptography (e.g., BB84 QKD) primarily requires high purity ($P$) and brightness ($B$) to minimise multi-photon emission and maximise secure key generation rates. Since security in QKD does not rely on photon interference, indistinguishability is not a critical factor.  
    \item Quantum computing and photonic quantum circuits, on the other hand, require high indistinguishability ($I$) and purity ($P$). This is because these applications rely on multi-photon interference effects, where distinguishable photons degrade quantum coherence and computational accuracy \cite{Bogaerts2020}.  
\end{itemize}



\subsection{TO be included somewhere}


Despite extensive efforts to engineer reliable single-photon sources (SPSs) \cite{Eisaman2011}, no simple and stable source has yet been developed with the necessary properties for practical QKD implementation. As a result, attenuated lasers are commonly used to generate weak coherent states \cite{Stucki2005}. However, these sources follow a Poissonian photon number distribution, meaning there is always a nonzero probability of emitting multi-photon pulses, leaving them vulnerable to photon-number splitting (PNS) attacks. While security-enhancing protocols, such as decoy state QKD \cite{Lo2005}, have been introduced to mitigate this risk, recent studies have demonstrated that even imperfect sub-Poissonian photon sources can outperform state-of-the-art weak coherent state lasers in secure quantum communication \cite{Ordan2024}.
