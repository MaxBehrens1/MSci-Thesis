\section{Introduction}

\begin{itemize}
    \item Why do we need single photons for quantum technologies 
    \begin{itemize}
        \item Communication (QKD)
        \item Sensing
        \item Metrology
    \end{itemize}
    \item Probabilistic vs. deterministic single photon sources
    \begin{itemize}
        \item $\chi^{(2)}$ in SPDC
        \item FWM
        \item Atoms (artificial \& natural)
    \end{itemize}
    \item Photonic trinity: PIB
    \begin{itemize}
        \item P: $g^{(2)}(t)$ \& HBT
        \item I: $T_2$, photon coalescence, HOM effect
        \item B: Purcell effect, cavity, reduced $T_1$  
    \end{itemize}
    \item hBN Description
    \begin{itemize}
        \item Discuss $T_1$ times seen experimentally
        \item Quantum efficiency paramter $\eta_{QE}=\frac{\gamma_{rad}}{\gamma_{rad} + \gamma_{non-rad}}$
        \item Debye-Weller effect $\eta_{DW}=\frac{I_{ZPL}}{I_{ZPL}+I_{PSB}}$
        \item Spectral purity $\Delta \nu \propto \frac{1}{T_2}=\frac{1}{T_2^*}+\frac{1}{2T_1}$. $T_2$ is the coherence time, $T_2^*$ is the pure dephasing rate and $T_1$ is the transition lifetime.
        \item Discuss $T_2$ measured experimentally i.e. Aymeric Delteil and Clarisse Fournier
        \item How to generate these defects (Annealing or Irradiation)
        \item Categorising defects (internet hBN-info)
    \end{itemize}
\end{itemize}

This work investigates the non-resonant excitation of hexagonal boron nitride (hBN) defects at room temperature to achieve bright and pure single-photon emission. Additionally, Purcell enhancement is pursued by coupling the single-photon modes to an open Fabry-Pérot cavity.

Efficient single-photon production is critical for advancing quantum technologies, with applications ranging from enhancing measurement precision beyond classical limits \cite{Nagata2007, Vitelli2010} to enabling reliable transport and processing of analogue information in photonic circuits \cite{Bogaerts2020}. Among these applications, quantum key distribution (QKD) is the most mature, leveraging the quantum properties of single photons to detect eavesdropping attempts and create secure communication channels. 

The first QKD protocol was proposed in BB84 \cite{Bennett2014}, in which Alice (the sender) transmits photons polarised in one of two bases: the rectilinear basis (horizontal, $0^{\circ}$, representing a 0, and vertical, $90^{\circ}$, representing a 1) or the diagonal basis (diagonal, $45^{\circ}$, representing a 1, and anti-diagonal, $135^{\circ}$, representing a 0). These states encode binary bits (0 or 1) in two mutually unbiased bases. For each photon, Bob (the receiver) randomly selects a basis for measurement (either rectilinear or diagonal) and, after detection, publicly announces which basis was used, without revealing the measured bit value. Alice and Bob then retain only the results where they chose the same basis, using them to construct a shared secret key.

If an eavesdropper (Eve) attempts to intercept and measure the photons, she must also randomly choose a basis. Since she does not know Alice’s choice, she has a 50\% probability of selecting the wrong basis for each photon. When this happens, the measurement collapses the quantum state into a definite polarisation, in the wrong basis. If Eve then resends the photon to Bob, it will now be in the incorrect state whenever she measured in the wrong basis. This disturbance introduces errors in the shared key, as Bob’s measurements will no longer perfectly correlate with Alice’s original encoding. When Alice and Bob compare a subset of their key bits to check for inconsistencies, any deviation from expected correlations reveals the presence of an eavesdropper. The probability of Eve measuring each photon in the correct basis by chance is $(\frac{1}{2})^n$, where $n$ is the number of bits compared. This probability decreases with $n$, meaning that security can be made arbitrarily strong by increasing the number of compared bits.

While BB84 is provably secure against eavesdropping in an idealised quantum setting \cite{Shor2000}, a significant weakness arises when multi-photon pulses are used instead of true single-photon sources. In such cases, Eve can exploit the photon-number splitting (PNS) attack by using a beam splitter to measure one of the transmitted photons while allowing the other to continue to Bob undisturbed. This allows her to extract information about the key without introducing detectable errors, highlighting the necessity of true single-photon sources for ensuring secure quantum communication.

Despite extensive efforts to engineer reliable single-photon sources (SPSs) \cite{Eisaman2011}, no simple and stable source has yet been developed with the necessary properties for practical QKD implementation. As a result, attenuated lasers are commonly used to generate weak coherent states \cite{Stucki2005}. However, these sources follow a Poissonian photon number distribution, meaning there is always a nonzero probability of emitting multi-photon pulses, leaving them vulnerable to photon-number splitting (PNS) attacks. While security-enhancing protocols, such as decoy state QKD \cite{Lo2005}, have been introduced to mitigate this risk, recent studies have demonstrated that even imperfect sub-Poissonian photon sources can outperform state-of-the-art weak coherent state lasers in secure quantum communication \cite{Ordan2024}.