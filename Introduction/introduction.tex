\section{Introduction}

\begin{itemize}
    \item Why do we need single photons for quantum technologies 
    \begin{itemize}
        \item Communication (QKD)
        \item Sensing
        \item Metrology
    \end{itemize}
    \item Probabilistic vs. deterministic single photon sources
    \begin{itemize}
        \item $\chi^{(2)}$ in SPDC
        \item FWM
        \item Atoms (artificial \& natural)
    \end{itemize}
    \item Photonic trinity: PIB
    \begin{itemize}
        \item P: $g^{(2)}(t)$ \& HBT
        \item I: $T_2$, photon coalescence, HOM effect
        \item B: Purcell effect, cavity, reduced $T_1$  
    \end{itemize}
    \item hBN Description
    \begin{itemize}
        \item Discuss $T_1$ times seen experimentally
        \item Quantum efficiency paramter $\eta_{QE}=\frac{\gamma_{rad}}{\gamma_{rad} + \gamma_{non-rad}}$
        \item Debye-Weller effect $\eta_{DW}=\frac{I_{ZPL}}{I_{ZPL}+I_{PSB}}$
        \item Spectral purity $\Delta \nu \propto \frac{1}{T_2}=\frac{1}{T_2^*}+\frac{1}{2T_1}$. $T_2$ is the coherence time, $T_2^*$ is the pure dephasing rate and $T_1$ is the transition lifetime.
        \item Discuss $T_2$ measured experimentally i.e. Aymeric Delteil and Clarisse Fournier
        \item How to generate these defects (Annealing or Irradiation)
        \item Categorising defects (internet hBN-info)
    \end{itemize}
\end{itemize}

This work investigates non-resonant excitation of hexagonal boron nitride (hBN) defects in an open, room-temperature Fabry-Pérot cavity to achieve bright, pure single-photon emission.

Efficient single-photon production is crucial for the development of many quantum technologies. They have been used to enhance measurement precision beyond classical limits \cite{Nagata2007, Vitelli2010} and integrated into photonic circuits for transporting and processing analogue information \cite{Bogaerts}.