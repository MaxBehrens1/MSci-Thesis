\section{Introduction}

\begin{itemize}
    \item Why do we need single photons for quantum technologies 
    \begin{itemize}
        \item Communication (QKD)
        \item Sensing
        \item Metrology
    \end{itemize}
    \item Probabilistic vs. deterministic single photon sources
    \begin{itemize}
        \item $\chi^{(2)}$ in SPDC
        \item FWM
        \item Atoms (artificial \& natural)
    \end{itemize}
    \item Photonic trinity: PIB
    \begin{itemize}
        \item P: $g^{(2)}(t)$ \& HBT
        \item I: $T_2$, photon coalescence, HOM effect
        \item B: Purcell effect, cavity, reduced $T_1$  
    \end{itemize}
    \item hBN Description
    \begin{itemize}
        \item Discuss $T_1$ times seen experimentally
        \item Quantum efficiency paramter $\eta_{QE}=\frac{\gamma_{rad}}{\gamma_{rad} + \gamma_{non-rad}}$
        \item Debye-Weller effect $\eta_{DW}=\frac{I_{ZPL}}{I_{ZPL}+I_{PSB}}$
        \item Spectral purity $\Delta \nu \propto \frac{1}{T_2}=\frac{1}{T_2^*}+\frac{1}{2T_1}$. $T_2$ is the coherence time, $T_2^*$ is the pure dephasing rate and $T_1$ is the transition lifetime.
        \item Discuss $T_2$ measured experimentally i.e. Aymeric Delteil and Clarisse Fournier
        \item How to generate these defects (Annealing or Irradiation)
        \item Categorising defects (internet hBN-info)
    \end{itemize}
\end{itemize}

This work investigates the non-resonant excitation of hexagonal boron nitride (hBN) defects at room temperature to achieve bright and pure single-photon emission. Additionally, Purcell enhancement is pursued by coupling the single-photon modes to an open plano-concave Fabry-Pérot cavity.

Efficient single-photon production is critical for advancing quantum technologies, with applications ranging from enhancing measurement precision beyond classical limits \cite{Nagata2007, Vitelli2010} to enabling reliable transport and processing of analogue information in photonic circuits \cite{Bogaerts2020}. Among these applications, quantum key distribution (QKD) is the most mature, leveraging the quantum properties of single photons to detect eavesdropping attempts and create secure communication channels. 

The first QKD protocol proposed was BB84 \cite{Bennett2014}, in which Alice (the sender) transmits photons polarised in one of two bases: the rectilinear basis ($\ket{H}$, and $\ket{V}$) or the diagonal basis ($\ket{D}=(\ket{H}+\ket{V})/\sqrt{2}$, and $\ket{A}=(\ket{H}-\ket{V})/\sqrt{2}$). These states encode binary bits (0 or 1) in two mutually unbiased bases. For each photon, Bob (the receiver) randomly selects a basis for measurement (either rectilinear or diagonal) and, after detection, publicly announces which basis was used, without revealing the measured bit value. Alice and Bob then retain only the results where they chose the same basis, using them to construct a shared secret key.

If an eavesdropper (Eve) intercepts and measures a photon, she must randomly select a basis, with a $ \frac{1}{2} $ probability of choosing incorrectly. When she measures in the wrong basis, the quantum state collapses into an incorrect polarization. If she resends the photon, it remains in the wrong state with probability $ \frac{1}{2} $, introducing an error in Bob’s measurement. Alice and Bob detect Eve by comparing $ n $ bits of their shared key. The probability of Eve measuring each photon in the correct basis by chance is $\left(\frac{1}{2}\right)^n$, which decreases exponentially with $n$, allowing security to be arbitrarily strengthened.

While BB84 is provably secure against eavesdropping in an idealised quantum setting \cite{Shor2000}, a significant weakness arises when multi-photon pulses are used instead of true single-photon sources. In such cases, Eve can exploit the photon-number splitting (PNS) attack \cite{Ashkenazy2024} by using a beam splitter to measure one of the transmitted photons while allowing the other to continue to Bob undisturbed. This allows her to extract information about the key without introducing detectable errors, highlighting the necessity of true single-photon sources for ensuring secure quantum communication.

\subsection{Benchmarking Single-Photon Sources}

An ideal single-photon source should emit exactly one photon on demand, with each photon possessing identical characteristics. To assess the quality of a single-photon source, three key parameters are evaluated: Purity ($P$), Indistinguishability ($I$), and Brightness ($B$). For an ideal source, all three parameters reach their maximum value, such that $ P = I = B = 1 $.

\subsubsection{Purity}
Purity is a loss-independent measure that quantifies the likelihood that the source emits only one photon at a time. It is defined as $P = 1 - g^{(2)}(0)$, where $g^{(2)}(0)$ is the second-order correlation function at zero time delay. The function $g^{(2)}(\tau)$ describes the statistical correlation between detecting two photons separated by a time delay $\tau$ relative to a completely random (Poissonian) photon distribution:

\begin{equation}
    g^{(2)}(\tau) = \frac{\langle I(t)I(t+\tau)\rangle}{\langle I(t)\rangle^2},
    \label{eqn:g2}
\end{equation}

where $\langle \rangle$ denotes a time-averaged quantity, and $I(t)$ is the measured photon intensity at time $t$. For an ideal single-photon emitter, $g^{(2)}(0) = 0$, indicating perfect antibunching, meaning that photons are emitted strictly one at a time, with no possibility of multi-photon events. In contrast, coherent light (such as from a laser) follows Poissonian statistics and has a $g^{(2)}(0) = 1$ (\textbf{Potentially show this mathematically}), which indicates no correlation between photon detections. A value of $g^{(2)}(0) > 1$ is characteristic of bunched light, such as thermal sources, where photons tend to arrive in clusters due to bosonic bunching effects. In practice, achieving $g^{(2)}(0) < 0.5$ is a strong indicator of a non-classical light source, supporting potential single-photon emission. This measurement is typically performed using a Hanbury Brown and Twiss (HBT) interferometer \cite{Brown1956}, which will be discussed in detail in \textbf{Section X}. 

\subsubsection{Indistinguishability}

Indistinguishability, also a loss-independent measure, quantifies the likelihood that two photons will interfere in a Hong-Ou-Mandel (HOM) setup \cite{Hong1987}. The Hong-Ou-Mandel effect states that when two indistinguishable photons arrive simultaneously at the input ports of a 50:50 beam splitter, they will always exit together through the same output port, rather than taking separate paths \cite{Hong1987}.

The degree of indistinguishability of a single-photon source is typically assessed by measuring the $g^{(2)}(0)$, often denoted $g^{(2)}_{\text{HOM}}(0)$, at the output of a path-unbalanced Mach-Zehnder interferometer \cite{Rarity2024}. One arm of the interferometer includes a delay line matched to the temporal separation between sequential photons. To achieve a perfect temporal overlap of consecutive photons, a pulsed laser, typically with a pulse width on the order of 3 ps, much shorter than the temporal envelope of the single photons (which is typically several nanoseconds at room temperature), is used for alignment. The laser is passed into the interferometer and one of the output ports of the second beam splitter is sent to a spectrometer. When the laser pulses are not temporally overlapped, a spectral beating pattern is observed due to partial interference. In contrast, perfect temporal alignment results in a Gaussian-shaped spectral feature.

The indistinguishability can be approximately expressed as $ I \approx 1 - 2g^{(2)}_{\text{HOM}}(0) $. This relation is only approximate because it assumes the source emits exactly one photon per pulse. In practice, multi-photon emission and other imperfections can affect the measured indistinguishability. The expression becomes exact in the ideal case of a perfect single-photon source with no multi-photon contributions. The factor of two arises from the fact that, for a pure but fully distinguishable photon source, $ g^{(2)}_{\text{HOM}}(0) = 0.5 $, as the two photons have a 50\% probability of exiting the same output port of the beam splitter due to the absence of quantum interference. 


For two photons to be indistinguishable, they must occupy the same spatial mode, have identical spectral and temporal profiles, and be in equivalent polarisation and phase states \cite{Senellart2017}. The spatial mode refers to the spatial distribution of the photon wavepacket; typically, single-mode optical fibres are used to ensure that photons are collected into the same well-defined optical mode. Spectral indistinguishability is determined by the line-width of the emitter, while the temporal profile is governed by the emitter’s lifetime, $T_1$. Polarisation indistinguishability requires that the photons have the same polarisation orientation, which can be ensured using polarisers and wave plates. The phase coherence of single photons is influenced by the pure dephasing rate, given by $1/T_2^*$, which can be measured using a Michelson interferometer \cite{Michelson1887, Jelezko2003}. A further explanation of the Michelson interferometer and the measurement of pure dephasing time, $T_2^*$, will be provided in \textbf{Section X}. 

The timescale over which an emitter produces indistinguishable photons is known as the coherence time, $T_2$, and is given by:

\begin{equation}
    \frac{1}{T_2}=\frac{1}{T_2^*} + \frac{1}{2T_1}.
    \label{eqn:coherence_time}
\end{equation}

 The coherence time determines the spectral linewidth of the emitter, as governed by the energy–time uncertainty principle. In the absence of dephasing mechanisms, the linewidth is said to be Fourier-limited and is determined solely by the emitter's lifetime. However, in realistic environments, dephasing processes such as phonon interactions introduce additional broadening, resulting in a linewidth greater than the Fourier limit.



\subsubsection{Brightness}

The brightness describes the probability that a single photon is emitted per excitation pulse. Unlike $P$ and $I$, brightness is a loss-dependent measure, as it depends on the detection efficiency and optical setup. Therefore, when comparing sources, it is important to distinguish between the brightness at the source ($B_s$), measured before any optical losses, and the detected brightness ($B_d$), which is affected by transmission losses, collection efficiency and detector efficiency. For meaningful comparisons, experimental setups must carefully characterise and account for losses throughout the optical system.

The brightness of a single-photon source is influenced by the emitter’s radiative lifetime, $ T_1 $, with shorter lifetimes enabling higher emission rates and thus brighter sources. This lifetime can be reduced by enhancing the spontaneous emission rate via the Purcell effect \cite{Anonymous1946}. By increasing the Purcell factor, $ F_P $, of the system the lifetime is reduced and given by $ T_1' = T_1 / F_P $. The magnitude of Purcell enhancement depends on both the properties of the cavity and the emitter, and is given by:


\begin{equation}
    F_P = \frac{3}{4\pi^2}\left(\frac{\lambda_{free}}{n}\right)^3\frac{Q}{V}, 
    \label{eqn:Purcell_factor}
\end{equation}

where $\lambda_{free}$ is the vacuum wavelength of the photons, $n$ is the refractive index of the cavities material and $Q$ and $V$ are the quality factor and volume of the cavity respectively. The quality factor, or $Q$-factor, of a cavity quantifies how well the cavity stores energy, and is defined as the ratio of the cavities resonant frequency to the linewidth of the emitter. A further discussion on the $Q$-factor will be given in \textbf{section X}.

Examples of emitters exhibiting Purcell enhancement include semiconductor quantum dots embedded in a micro pillar cavities \cite{Engel2023}, and organic dye molecules placed in the near field of a specially designed plasmonic nanoantenna \cite{Zhao2020}. 

\subsection{Probabilistic vs. Deterministic Single-photon Source}

Probabilistic single-photon sources based on nonlinear optical processes, such as spontaneous parametric down-conversion (SPDC) and four-wave mixing (FWM), have historically been the most widely used methods for generating single photons. These sources are termed probabilistic because, for each pump pulse, there is a probability of producing zero, one, or multiple photon pairs. When only one pair is produced, the detection of one of the photons, called the idler photon, will herald the presence of the second photon, which can be used as the signal. 

The state of $n$ photon pairs can be described using a two mode squeezed vacuum state

These sources are termed \textit{probabilistic} because, for each pump pulse, there is a certain probability of producing zero, one, or multiple photon pairs. When exactly one pair is created, detection of one photon (the \textit{idler}) can herald the presence of its partner (the \textit{signal}), thereby generating a heralded single photon. A simple model for the quantum state of light generated by such sources is the two-mode squeezed vacuum state:
\[
|\psi\rangle = \sqrt{1 - |\lambda|^2} \sum_{n=0}^{\infty} \lambda^n |n,n\rangle,
\]
where \( |\lambda|^2 = \tanh^2(r) \), and \( r \) is a dimensionless squeezing parameter proportional to the pump power. The probability of generating exactly one pair \( |1,1\rangle \) increases with the pump strength and ideally follows \( B = \tanh^2(r)/\cosh^2(r) \), where \( B \) denotes the brightness.

However, as the pump power increases, so too does the likelihood of generating higher-order terms \( |n,n\rangle \) for \( n > 1 \), which can lead to multi-photon contamination. If photon-number-resolving detectors are unavailable, or if photon loss occurs, the detection of an idler photon may correspond to more than one photon being present in the signal arm, reducing the purity \( P \). In the worst-case scenario, the purity is reduced to \( P = 1 - 2\tanh^2(r) \). To avoid this degradation, operation is typically limited to lower pump powers (small \( r \)), which in turn limits the brightness.

Recent advances have focused on improving detector efficiency and photon-number resolution to reject multi-pair events and thereby increase \( P \) for a given \( B \). Nonetheless, even under ideal conditions, where all multi-photon events are eliminated, the maximum achievable brightness from probabilistic sources is fundamentally limited. For instance, with a squeezing of \( r = \log(\sqrt{2} - 1) \), the upper bound on \( B \) is just \( 1/4 \). Achieving higher brightness values (e.g. \( B = 1 \)) using such sources requires multiplexing multiple probabilistic sources in a synchronised network—a technically demanding approach that introduces additional complexity and loss, which can again degrade \( B \) and \( P \).


\newpage

Despite extensive efforts to engineer reliable single-photon sources (SPSs) \cite{Eisaman2011}, no simple and stable source has yet been developed with the necessary properties for practical QKD implementation. As a result, attenuated lasers are commonly used to generate weak coherent states \cite{Stucki2005}. However, these sources follow a Poissonian photon number distribution, meaning there is always a nonzero probability of emitting multi-photon pulses, leaving them vulnerable to photon-number splitting (PNS) attacks. While security-enhancing protocols, such as decoy state QKD \cite{Lo2005}, have been introduced to mitigate this risk, recent studies have demonstrated that even imperfect sub-Poissonian photon sources can outperform state-of-the-art weak coherent state lasers in secure quantum communication \cite{Ordan2024}.
